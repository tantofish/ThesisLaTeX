\begin{abstractEN}
In this thesis, we propose a system to estimate head poses only using depth information in real-time. Two methods are developed. First, assuming that a head can be approximated by a bounding box, we find the best fitted plane for the frontal face by the least square error method. Thus, the normal vector of this plane represents the head orientation. Second, an optimization method based on 3D model fitting is developed. We iteratively minimize the distance between source and target point clouds of a user's head. This method is more robust and the results are more precise. Both of the proposed methods give fully real-time responses (30fps) without needing the GPU speedup. We adopt a commodity depth sensor named Microsoft Kinect as well as Asus Xtion, and use the depth image as the only input so that our system will not be affected by illumination variations. The simplicity of this acquisition device comes at the cost of frequent noises in the acquired data. We demonstrate that 6 degree of freedom real-time head motion tracking in 3D space can be achieved with noisy depth data.

 

\noindent\bf{Keywords: Real-Time Head Motion Tracking; Depth Image; Kinect; 3D Template Matching; Iterative Optimization; Least Square Method}

\end{abstractEN}

\begin{comment}

%\category{I2.10}{Computing Methodologies}{Artificial Intelligence --Vision and Scene Understanding} \category{H5.3}{InformationSystems}{Information Interfaces and Presentation (HCI) -- Web-based Interaction.}

%\terms{Design, Human factors, Performance.}

\keywords{Head Pose Estimation \and Depth Map \and Kinect \and Least Square Error Plane \and Optimization \and Gradient Decent Algorithm \and Nose Tracking}

\end{comment}
