\begin{abstractCH}

\setlength{\baselineskip}{1.5em}
在本篇論文中,我們對於以深度影像為基礎達成即時頭部軌跡追蹤之問題提出了兩種不同的方法。在方法一中,我們在深度圖上對使用者的正臉取樣產生點雲(Point Cloud),並使用這些點雲以最小平方法計算出最佳近似的平面,再將臉部輪廓以最小平方法計算出最佳近似的橢圓,以此平面之法向量及橢圓之傾斜角計算出頭部之旋轉角度。在方法二中,我們使用梯度坡降法迭代地將特定距離函數進行最佳化,此方法可算出更精準的旋轉角度及移動距離。我們所提出的這兩個方法,在單一CPU運算下皆可達到30fps的計算速度。在此系統所使用之攝影器材─Microsoft Kinect以及Asus Xtion Pro,乃較為普及、平價、易取得且易使用的深度攝影機,其優點亦伴隨著較高的拍攝雜訊。為了使系統得以不被環境光線改變所影響,我們只利用其拍攝出之深度影像為輸入來源做計算。在本篇論文中,我們證明,三維空間中,六個自由度的即時頭部軌跡追蹤,是可以經由分析帶有雜訊之深度影像達成的。

 

\noindent\bf{關鍵字: 即時頭部運動軌跡追蹤; 深度影像; 三維樣板比對; 迭代最佳化; 最小平方法}
\end{abstractCH}
