\begin{acknowledgementsCH}

\setlength{\baselineskip}{1.5em}
口試結束了!學生生涯也快結束了!

最後一兩個月,起初每天都被壓力壓到喘不過過氣,每天都想著什麼時候能結束?直到最後一個禮拜,我才意識到所剩的時間不多,既然這都是學生生涯的Last Shot了,有什麼理由不讓他變成Best Shot?於是用力的提起勁來做事。想著要給老師驚喜,也想著不要讓老師失望,也不想讓自己後悔。很高興最後我沒有留下遺憾。最重要的是我覺得沒有愧對老師,也沒有愧對自己!

一路走來,真的很感謝老爸以及家人的支持。感謝小ㄅㄟˊ在所有事情上面的體諒,妳的體貼是支持我最大的動力。感謝一路走來的戰友們楊證諺、大黃、小黃、阿崴、明鑫、土涵、田波、阿璽、痞子,無論是歡笑出遊或是痛苦熬夜,都很榮幸能跟你們一起度過。感謝特地捎來祝褔的摳拉、鳥鳥、大喬,你們的祝福讓我多一分勇氣站在台上。很感謝陪我一起在第一戰線被老師念、又陪我在最後戰線在R544待到天亮的小捲。感謝葉老大這位兩年來數不清次數在關鍵時刻給我幫助的靠山。很感謝大黃還有學弟妹Larry、Jean Wang、何佳儒口試時辛苦幫忙食物飲料!很感謝鐵哥,每次跟你聊真的都學到很多。感謝許碩傑、黃鈞愷,你們永遠是我最敬愛的學長。最感謝的是老師、每次Meeting都給我不同的啟發、困惑的時候都給我明確的方向,還有感謝老師對我的信任,非常感謝老師當初願意指導我,我真的覺得很幸運,謝謝老師!

最後寫給還沒口試的好朋友們,如果你們看到這篇文,不要生氣覺得我那麼快就開始寫畢業感言,而是看到前面說的:「最後三四天,才驚覺時間不多」。當研究生,難的就是要相信,相信自己走在對的路上,相信自己正在做對的事情,看不到終點的比賽最難完成,如果你自己都不能相信現在正在做的是對的事情,那短短的"printf();"都可以敲一分鐘。請不要懷疑自己,當你「相信」自己能畢業,下一階的問題就是,我要以什麼姿態畢業,我要畢業的得過且過還是讓自己驕傲?當問題從被動變成主動,寫Code也會從被動變成主動!你們個個都比我強不只一點點,我都把你們當神在拜了,要相信自己呀!加油!



\end{acknowledgementsCH}
